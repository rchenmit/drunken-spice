% This is paper.tex

%\title{
%\vspace{-2cm}
%What is sex? Evolution of a traditional concept as a result of scientific research and increased public awareness\\
%\vspace{-.5cm}
%}
%
%\author{
%Robert Chen\\
%\vspace{-1.6cm}
%21A.355 - Anthropology of Biology; Paper 2\\
%\vspace{-.0cm}
%Massachusetts Institute of Technology
%%\vspace{-.2cm}
%}
%\date{
%\today
%}
%
%\maketitle
Performance enhancing drugs have been used by athletes of all ages and skill levels.  Specifically, anabolic steroids (anabolic steroids) are of major interest to these athletes, because of their ability to help build muscle in a way that would be unachievable through natural means. Because of ethical concerns regarding the fairness and safety of athletes using anabolic steroids, these substances are controlled heavily. Furthermore, the are outlawed in several sports such as baseball, track and field, and football as well as all Olympic events. The biosociality and biocapitalism surrounding the usage of steroids revolves around the idea of nature. Certain people believe that the usage of such substances renders a human body unnatural. It is interesting to compare the issues surrounding the circulation of anabolic steroids, to those surrounding the circulation of seemingly less stigmatized technologies, such as cognitive enhancement drugs, or even things as simple as contact lenses. In this paper, we explore the regulation and social issues surrounding the usage anabolic steroids in professional sports, compared with less stigmatized technologies that work to alter the natural human condition.  
	
Anabolic steroids are drugs that mimic the effects of the male sex hormones testosterone and dihyrdotestosterone, both of which are steroid compounds. \cite{nida} Steroids in general are a large class of organic compounds that exist in a 17-carbon, 4-ring system. \cite{mw}The sex hormones testosterone and estradiol are examples of natural steroids. Anabolic steroids help anabolism, which is caused by the heightened protein synthesis that occurs. Furthermore, anabolic steroids result in androgenic properties, or properties pertaining to  male secondary sex characteristics. anabolic steroids also induce virilization, which is the biological development of sex differences. Other ergogenic effects that are provided by anabolic steroids include increases in bone mineral density, glycogen storage, lipolysis, neural transmission, muscle endurance, strength and power, as well as enhanced recovery from injury, and behavior modification (aggression). On the other hand, various adverse effects are associated with anabolic steroids, including decreased myocardial function, gynecomastia, reduced sperm count (in males), clitoreomegaly (in females), libido changes, acne, risk of liver damage, and psychological effects including mania, depression, aggressiona and mood swings. \cite{hoffman}
	
Anabolic steroids have also been used to treat disease such as anemia, breast cancer, and angioedema. Anabolic steroids are effective for these purposes as they help to rebuild lost tissue. \cite{mayo_ana}
	
The road to the development of steroids was started with the discovery that testosterone was absorbed through the blood as with other hormones. In 1849, scientist Arnold Berthold removed the testicles from chickens, and noticed that the chickens exhibited diminished sexual characteristics as well as less aggressive behavior compared to non-castrated chickens. When Berthold implanted testes from other birds into the original chickens' abdomens, it was noticed that the chickens exhibited no change in virility, and the original chickens exhibited normal development of secondary sex characteristics. Furthermore, these chickens exhibited increased capillarization in the body, it was suggested that the testes were secreting a substance into the bloodstream, which would in turn lead to downstream development of sexual characteristics. The substance was later deemed the hormone testosterone, and later efforts to synthesize anabolic steroids would be based upon the production of this compound. \cite{mehta}
	
The use of testosterone to enhance performance in sports began as early as the 1940s. \cite{anabolex} During the 1940s and 1950s, “testosterone compounds were experimented with by some west coast bodybuilders”. \cite{yesalis} The first dramatic case of anabolic steroid usage was during the 1954 Olympics, when weightlifters were discovered to have used them for performance enhancement. Since then, anabolic steroids have been used widely by power lifters, National Football League players, Major League Baseball players, as well as collegiate athletes. \cite{hoffman} Famous athletes who have been reported to have used steroids include Olympic gold medalist sprinter Marion Jones and Major League Baseball Players Mark McGuire and Sammy Sosa.

	
Ben Johnson's Olympic gold finish in the 100 meter  event of the 1988 Olympics in Seoul sparked controversy when he tested positive for anabolic steroids. This prompted the U.S. Congress to consider regulating anabolic steroids under the Controlled Substances Act. In 1990, the Anabolic Steroid Control Act of 1990 first deemed anabolic steroids as controlled substances.  The Anabolic Steroid Control Act of 2004 added prohormones (substances that could be absorbed into the body and later converted to testosterone) to the list of controlled substances. \cite{checkana}
	
Such political movements that regulate anabolic steroids probably stem from both a combination of issues regarding the fairness of the competition in sports as well as safety concerns that are uncovered as more knowledge about the biological effects of anabolic steroids are uncovered. 

Although anabolic steroids remain as controlled substances in America, there is demand for them among certain demographics and therefore the circulation of anabolic steroids prospers in the underground markets. For example, a study showed that in Kansas and Missouri, $54\%$ of the study's mail subjects and $10\%$ of female subjectswere using steroids on a regular basis. \cite{tricker} In 1987, the “black market in steroids [was] estimated at \$100 million a year, according to the FDA”. \cite{stehlin} It is not uncommon for these black market steroids to contain different ingredient makeups than what is found in actual anabolic steroids. For example, a study done at the Institute of Forensic Medicine in Germany showed using mass spectrometry and gas chromatography that in $35\%$ of the drugs bought on the black market used for their study did not contain the expected ingredients.  Since anabolic steroids are controlled for purposes of safety, they do not experience the same issues regarding biocapital as would technologies such as genomics. 

An interesting issue that is raised by the anabolic steroid issue is the distinction between what is natural and what is not.  Some people may claim that any muscle mass gained with the aid of anabolic steroids would not be considered natural. This issue can be compared to that of genetically modified crops, which have generated mixed opinions among the public.  Genetically modified crops are considered an assault on nature by some, who make claims such as:  “[genetically engineering manipulates] the very fabric of life, shuffling the genetic deck that controls every aspect of every living organism in ways that nature never intended”. \cite{grogan} This is similar to the notion that anabolic steroids allow users to build muscle in ways that their bodies were never intended to be able to. Crops are often selectively bred in order to create new genetic variation and enhance desirable traits in pants. Some new traits that have been introduced into plants include: fungal disease resistance in corn, altered fatty acid ratios in canola, nematode resistance in beets, and virus resistance in rice. \cite{massey} By comparing anabolic steroid usage in humans, with genetic engineering in plants, notice that an organism should not be considered natural given that it's chemical makeup has been changed in some way.

Ethical issues are continually raised by biologists regarding the genetic modification of crops. 
A critic of the famous plant breeder Luther Burbank claims in 1906 that “we have recently advanced our knowledge of genetics to the point where we can manipulate life in a way never intended by nature. We must proceed with utmost caution in the application of this new-found knowledge”. \cite{massey} this underscores both the power associated with genetic engineering as well as the potential situations which could go awry when modifying crops. For example, with genetic engineering it is now possible to make tomatoes which is up to 90 times larger than the ancestral wild tomato, with a larger percentage of the tomato consisting of pulp, the nutritious aspect of tomatoes. \cite{massey}
	
Tim Delaney and Tim Madigan believe that genetic engineering actually may play a role in performance enhancement for sports in the future. Currently, there are no strict regulations regarding genomic technology, likely because not much is known about genomics to date. \cite{delaney}But if genetic engineering can be used to modify crops to make them sweeter, bigger, fuller, and more appealing to the eye, who's to say that genetic engineering cannot be used to enhance muscle growth? With the potential application of genetic engineering towards anabolic processes, new methods for detection of such forms of “cheating” would need to be developed as well, likely at the DNA level.  This may be near impossible to perform, and if so there would be no feasible way to detect such forms of performance enhancement.  
	
At any rate, future biocapital in sports medicine could very well be grounded in genomic technology.  There is already tremendous promise for regenerative medicine, which aims for stem cell implants to grow into fuller tissues in the body. Such technology would probably initially be used for injury recovery, but could also be used for muscular hypertrophy purposes. 
	
How is something as commonplace as a sport, suddenly soiled by minor alterations in the body that deem people “unnatural”? People who oppose the usage of anabolic steroids claim that it is unfair for athletes to gain an unfair advantage by using anabolic steroids when competing against opponents who remain “clean”. Another issue at hand is that accomplishments made with the aid of anabolic steroids do not measure true athletic ability. More severe reasons include that steroid usage encourages further types of deviant behavior, “by showing that people who cheat can still become successful”. \cite{delaney} Although the usage of steroids can be compared to genetic engineering in that both may exhibit unnatural characteristics, anabolic steroid usage is typically framed in a much more stigmatized context because of the moral violations associated with it.
	
The usage of anabolic steroids in athletes can be mirrored by the usage of cognitive enhancement drugs in the academic profession.  In both cases, performance enhancement is used in the form of drugs. In order to help themselves concentrate on schoolwork, many students take drugs like Ritalin, which are normally prescribed for children with attention deficit disorder.  An even more common example is the usage of caffeine pills, and beverages like coffee, Red Bull, or Monster which contain various brain stimulants that help to increase alertness. 
	
Doping in academia may very well be just as widespread as the usage of performance enhancing drugs in sports.  In a documentary in the journal Nature, two Cambridge University researchers reported that “about a dozen of their colleagues had admitted to regular use of prescription drugs like Adderall, a stimulant, and Provigil, which promotes wakefulness, to improve their academic performance”. \cite{carey} Like anabolic steroids, these medications serve to treat diseases in impaired individuals. Adderall is used to treat attention deficit hyperactive disorder and Provigil is normally used to treat narcolepsy. One could argue that, these academics are enhancing themselves beyond what is considered a natural state, as they would be able to attain mentally what would not be possible otherwise.
	
Although both anabolic steroids and cognitive enhancement drugs are used for performance boosting beyond the natural state, the usage of anabolic steroids is much more stigmatized than the analogous case involving cognitive enhancement drugs. For example, former Major League Baseball player Barry Bonds' reputation was tarnished when it was confirmed that used anabolic steroids during his career. People wanted an asterisk placed next to his name in the record books, signifying that while he broke Hank Aaron's record for number of career home runs, he did not accomplish this in a natural human state.  Similar public eruption occurred when former St. Louis Cardinals player Mark McGuire was found to have used anabolic steroids during the 1998 season when he broke the record for the greatest number of home runs per season. But why don't people insist that the analogous asterisks be attached to the names of Nobel Laureates or Pulitzer Prize winners who have used forms of cognitive enhancement to help reach their goals?
	
While both cognitive enhancement drugs and anabolic steroids are controlled substances that require prescriptions in order to be obtained, there is likely less cultural awareness regarding cognitive enhancement in healthy individuals as compared with anabolic steroid usage in athletes. Furthermore, there may be a perception among the public that, while sports are simply games in which athletes compete, academics is a worthwhile endeavor that could bring many benefits to society. The public awareness regarding cognitive enhancement in healthy individuals, however, will likely change in the near future if certain stipulations such as mandatory usage for company employees are instigated. This would be analogous to requirements for military soldiers to take modafinil in order to stay awake for extended periods of time.
