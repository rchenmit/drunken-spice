% This is paper.tex

%\title{
%\vspace{-2cm}
%What is sex? Evolution of a traditional concept as a result of scientific research and increased public awareness\\
%\vspace{-.5cm}
%}
%
%\author{
%Robert Chen\\
%\vspace{-1.6cm}
%21A.355 - Anthropology of Biology; Paper 2\\
%\vspace{-.0cm}
%Massachusetts Institute of Technology
%%\vspace{-.2cm}
%}
%\date{
%\today
%}
%
%\maketitle

What is sex? In the medical environment, “sex” is usually defined in one of several ways. According to the Merriam-Webster dictionary “sex” is defined as “either of the two major forms of individuals...female or male [based on their] reproductive organs and structures”. \cite{webster} In this case, sex is used as a means of classification.  Sex can also represent the mixing of genetic material (as in sexual reproduction).  These two perceptions of the notion of sex have predominated in people's minds throughout history. However, current developments in biological research and biosociality make the issue more complex as they provide the public more insight on the presence of intersexed people. Furthermore, research on alternative means of creation such as cloning and in vitro fertilization raise new questions about what exactly constitutes sex.


Inherently, human beings like to subdivide things into categories. We do it with foods -- bananas, apples, and oranges are collectively referred to as fruits. Cucumbers, broccoli, and carrots belong in the ''vegetable" category. All people with a vagina are called ''female" and all people who possess a penis are called ''male".  Grouping items into bins is convenient for us. In the case of males vs. females, most people saw no significant problems with such a classification system, simply because they were not aware of a strong presence of intersex people. 

	Perhaps people are not aware about the phenomenon of intersexuality because usually when a hermaphroditic baby is born, there is a strong pressure to perform surgery to morph the newborn into a certain, definite male or female sex. Thus, the number of intersex individuals who actually remain intersex for purposes of categorization is small. In the past in Western European societies, the “hermaphrodite” category of sex was recognized.  However, “medicine gradually appropriated to iteself the authority to interpret -- and eventually manage – the category which had previously been widely known as 'hermaphroditism'. Victorian medical taxonomy began to efface hermaphroditism as a legitimated status by establishing mixed gonadal histology as a necessary criterion for 'true' hermaphroditism”. \cite{chase} 

	Much of the push to change intersexed bodies to those resembling a single sex came from the medical community. In the 1920s and 1930s, Hugh Hampton Young of the Johns Hopkins University developed various surgical techniques for transformation of intersexed bodies, with the effort to “normalize their bodily appearances to the greatest extents possible”. \cite{chase} This motivation has stuck with doctors since then, and thus, intersexual-born infants are deemed “psychological emergencies” that need to be corrected.  Some parents of these intersex children do not actually inform their children that such surgeries had been performed to them. After all, they thought, that their son or daughter would just grow up assuming that they were born that way, and if they had no reason to think otherwise then no major problems would result. Today, intersex surgeries happen quite often – about five times per day in the United States. \cite{libidomag}

$$++++$$

There exists some degree of societal pressure to morph intersex individuals into members of one of the two commonly known sexes. At this point in time, there is not enough awareness of intersex individuals and as a result it is difficult to incorporate them into society. For examples, restrooms are still only divided into male and female rooms. Which restroom would an intersex individual go to when stuck at an airport or a college campus? Should intersex individuals be allowed to join fraternities or sororities? People can support their acceptance, but the members of these Greek organizations would potentially be reluctant to invite intersex individuals to join their organization even though the intersex individuals may be otherwise qualified. Furthermore, how would intersex children be integrated into schools? An issue may arise in which other kids make fun of an intersex child and the intersex child becomes depressed or confused because of this. In an effort to keep their children from being confronted with such difficult issues, parents of intersex children may elect to morph the children into a specific sex, and to raise the children as if they were strictly female or strictly male.
	
Surgically altering intersexed individuals may seemingly provide them with a better life than they would have otherwise. However, sometimes intersex children who are raised as either male or female may not actually identify with that sex. There have been many cases of such children who desire to be identified as the other sex. For example, an person named “Steve” who was born intersexed  , was surgically altered at birth so that he exhibited male-like characteristics. Steve was actually born with the karyotype XY, the same karyotype that normal boys would be born with. However, Steve was born with a vagina as well as two genital abnormalities: severe hypospadias (opening of the urethra is on the underside of the penis rather than at the end of it) \cite{pubmed-hypo}and with a bi-fid scrotum (where the two halves of the scrotum meet above the penis) \cite{medscape}. Steve's surgeon sewed up Steve's labia and crafted a micropenis, making him into a boy without descended testes. \cite{libidomag}
	
Steve did not act the say other boys did, however. His peers in grade school called him a “faggot” because he acted like a girl. He wanted to be a girl, even though he lived his life as if he was a boy. He became depressed and contemplated suicide several times, and didn't realize that he was actually intersexed until he read through his medical history when he was over 40 years old. Steve ended up taking estrogen injections so that he could develop more feminine characteristics.  \cite{libidomag}

Steve's story shows that decisions to turn an intersexed child into a certain sex may not necessarily provide the child with a life that the child believes is acceptable.  Steve and other intersexed individuals have tried to speak out about their experiences as intersexed individuals, with the hope that the public will accept the intersex as a different category of sex, in addition to male and female. Cheryl Chase, an intersexed individual who had her clitoris surgically removed at birth, was a vanguard in biosociality regarding the intersex issues. To help spread public awareness of the intersex condition and to aid intersexed individuals in speaking out about their experiences, Chase started the Intersex Society of North America (INSA), a support group for intersexed individuals.  Among the INSA's goals include letting the public know about the prevalence of intersex births and urging parents to “resist medical pressure for unnecessary genital surgery and secrecy and to find their way to a peer-support group and counseling rather than to a surgical theater”. \cite{chase}
	
Such evidence that intersex children who have been surgically modified to fit a specific sex may not actually feel like they fit in that specific gender role indicates that it may be worthwhile to consider the intersex condition as a normal category of sex.  Sociologist Anne Fausto-Sterling actually proposes that there be five different sexes, each pertaining to what degree a person possesses reproductive traits of the traditional male or female. Fausto-Sterling's sexes are 1) “merms” - those who have aspects of female genitalia but do not have ovaries; 2) “ferms” - who have ovaries and male-like genitalia but no testes; 3) true hermaphrodites – who have the reproductive features of both sexes (both ovaries and testes); and 4 and 5) the traditional male and female. \cite{fausto} If society is able to accept such a concept, then intersex people may not have to feel the pressure to keep their condition a secret, for fear of being ridiculed. It might also lessen the pressure for parents of intersex children to push for surgical intervention. However, this would be easier said than done. The world would need to accommodate intersex people in a variety of situations, such as same-sex college dorm rooms and same-sex bathrooms.

The notion of sex as a means of classification has seen similar conflicts as with the notion of homosexuality. Traditionally, homosexuality was considered an immoral behavior choice, and homosexual individuals were condemned for such “immoral” behavior. The Old Testament mentions that homosexual behavior is considered an abomination and that “[homosexuals] shall surely be put to death; their blood shall be upon them” (English Standard Version, Leviticus 20:13 KJV). \cite{biblios20:13} However, recent research has discovered that homosexuality is not purely a behavioral choice but may be caused by natural biological processes as well. Biological processes such as genetic interactions and prenatal exposure to chemicals like amphetamines and nicotine are known to contribute to homosexual tendencies. \cite{savic} When considering the issue with sex as a means of classification, the general public believes that “male” and “female” are the normal conditions, while the “intersex” condition is abnormal.  Much like sex can actually be a gradient of phenotypes, ranging from purely male to purely female and including everything in between. Of course the pure males and pure females make up most of the population, but people may soon start to accept these in-between individuals as normal. Much like the classification of sexes, the notion of homosexuality as a naturally caused condition will be accepted more by the public as medical research develops on the subject. The main bottleneck, again, has to do with biosociality.

The classification of sexes poses similar issues as classification of human races. Like sex, race is another convenient form of classification. People are placed into different “race” categories which are predetermined by what society views as the main groups of people on earth. With such a categorization comes the notion that certain races are “better” than others. These notions of superiority over others fuel unnecessary biases towards or against certain groups. What is difficult about the classification of races is that it can be subjective. If a white person and a brown person have a child, what race would the child be considered? The only choices are brown and white. Biracial people may have trouble choosing a racial identity in times such as when applying for colleges or scholarships.
	
Up until the year 2000, Americans could only check one box for race on the yearly national Census. \cite{census} An article in TIME cites recent research stating that some half-black people may try to pass as black because these biracial people look more black than they do white (i.e. their skin is dark). Furthermore, it is argued that “biracial respondents pass as black ... to fit in with black peers in adolescence (especially since many claim that whites reject them)” (cited in \cite{time}). It is also argued that biracials try to pass as black in order to take advantage of affirmative action for purposes of winning more scholarship money or to gain educational and employment opportunities. \cite{time} The concept of race as a classification scheme saw some changes throughout time, as the public became more aware about multi-racial individuals. As with the issues of sex and sexual orientation as classification schemes, the concept of race changes due to biosociality. Increased public awareness changes traditional views in order to help accommodate individuals that do not fit into clearly defined categories.

$$++++$$

	Apart from its usage as a classification scheme, sex can be defined as the mixing of genetic material in order to reproduce, to create offspring. Traditionally, sexual reproduction between males and females was accepted as the standard of sex. This was the only known way to create a new organism. However, recent research challenges this notion, as new methods of creating life are discovered. Dolly the sheep represents a means of producing life through a combination of sexual and asexual reproduction. Dolly was produced by somatic cell nuclear transfer, in which genetic material was transferred into an egg cell and the cell was implated into a surrogate mother, who gave birth to the new individual.  Dolly was reffered to by some as a clone, since the genetic material used to create her was the same as that of another individual.  Dolly's creation changes the notion about sex; existing definitions and formations of it “have been, is a sense, sampled, remixed, resequenced, and provided with a novel means of amplification”, says sociologist Sarah Franklin. \cite{franklin} Essentially, the natural process of sex can now be manipulated and produced manually.  One can force two specific samples of genetic material to mix with each other in order to create a new individual that is desired to a certain specification. Franklin believes that Dolly represents the dawning of a new age ofgenomic technology in which such processes can happen: “[Dolly] is not, technically, a clone in terms of a part regenerated from a larger whole, for her origins lie in two cells that were merged, or mixed, to make her”. \cite{franklin}The development of new technologies, new ways of creating life, such as the method of somatic cell nuclear transfer used to create Dolly, challenges the original notion of sex as a means of creating life through mixing of genetic information between two individuals.

Sex has been defined by people as either, a classification of individuals based on their reproductive organs and genitals, or as the mixing of genetic material in order create a next of kin. New developments in scientific research as well as increased public awareness about intersexed individuals,  have led us to reconsider the meaning of the term ''sex''.  In fact, the complexities surrounding the notion of sex are paralleled in other hot topics such as homosexuality and race. The meaning of sex as natural means of creation of offspring is also challenged by new scientific research.  As the general public to delve into these issues, special people that don't fall into strictly defined categories of sex may be better accommodated in society.

