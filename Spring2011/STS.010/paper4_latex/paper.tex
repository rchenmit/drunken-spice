% This is paper.tex

\title{
\vspace{-2cm}
The development of understanding of antisocial personality disorder and its neuroscientifically-based treatments\\
\vspace{-.5cm}
}

\author{
Robert Chen\\
\vspace{-1.6cm}
STS.010 - Neuroscience and Society; Paper 4\\
\vspace{-.0cm}
Massachusetts Institute of Technology
%\vspace{-.2cm}
}
\date{
\today
}

\maketitle
	People with antisocial personality disorder are sometimes known as psychopaths, or sociopaths. It is widely believed that there are biological as well as non-biological causes for antisocial personality disorder. \cite{pickersgill}Before much was known regarding the biological causes of the disorder, the general public was tempted to label psychopaths as simply "crazy" people who did not contribute to society but rather harmed society. With the advent of various medical technologies, however, some biological aspects of the disorder are being explored in greater detail. Differences in brain imaging are being looked at, and genetic links to the disorder are being discovered. There has been an increasing amount of evidence that suggests a biological basis for antisocial personality disorder. Because such a biological basis is suggested to exist, it has been suggested that biologically based solutions can be used to change people with antisocial personality disorder to make them more accommodating to society.  In general, as more is known about the disorder, the state of psychopaths may be less stigmatized, as people move toward classifying the disorder as just another behavior difference with a known etiology, rather than a conscious life decision that leads one to a lifestyle that is destructive towards others. 
	The Diagnostic Statistical Manual of Mental Disoders defines antisocial personality disorder with the following characteristics:
A. There is a pervasive pattern of disregard for and violation of the rights of others occurring since age 15 years, as indicated by three (or more) of the following: 
    (1) failure to conform to social norms with respect to lawful behaviors as indicated by repeatedly performing acts that are grounds for arrest 
    (2) deceitfulness, as indicated by repeated lying, use of aliases, or conning others for personal profit or pleasure 
    (3) impulsivity or failure to plan ahead 
    (4) irritability and aggressiveness, as indicated by repeated physical fights or assaults 
    (5) reckless disregard for safety of self or others 
    (6) consistent irresponsibility, as indicated by repeated failure to sustain consistent work behavior or honor financial obligations 
    (7) lack of remorse, as indicated by being indifferent to or rationalizing having hurt, mistreated, or stolen from another 
Other criteria are that the individual is at least age 18 years,there is evidence of Conduct Disorder with onset before age 15 years, and that the occurrence of antisocial behavior is not exclusively during the course of Schizophrenia or a Manic Episode.

What should be emphasized (and is sometimes overlooked by people) is that the individual must have exhibited evidence of a conduct disorder before age 15, even though he or she may not have been diagnosed until age 18 or older. This is because, often times, individuals with antisocial personality disorder are not diagnosed until very severe behavioral manifestations of the disease are discovered, usually by other people. Examples of behavior that children under the age of 15 may exhibit including the starting of small fires (for example, with substances such as WD-40 and lighter fluid), cruelty to animals, or difficulty dealing with figures of authority such as parents or school teachers.  A common misconception that is held by people is that individuals with antisocial personality disorder are all criminals or murderers. \cite{yt-xbadaboom}This is not necessarily the case, as per the guidelines given above, other behaviors can characterize pyschopaths. Another common misconception is that only males exhibit antisocial personality disorder, when in fact antisocial personality disorder occurs in 3\% of males and 1\% of females. \cite{pickersgill}
	
Previously, people thought about antisocial personality disorder as a consequence of childhood nurture or upbringing. It was believed that the environment in which the child grew up in would be one of the greatest influences to the child's personality. Behavioral characteristics corresponding to antisocial personality disorder was often referred to clinically by doctors as psychopathy. In the 1950s psychopathy was diagnosed in individuals who exhibited criminalistic behavior but did not necessarily show other major mental handicaps. Martyn Pickersgill of the University of Edinburgh describes that psychopathy was an “ambiguous concept” back then, because the specific behaviors associated with it were not well defined. \cite{pickersgill} Furthermore, the public's opinion of psychopaths was influenced by literature at that time, which emphasized that although psychological causes contributed to psychopathic behavior, a major component of the cause of psychopathy was due to the environment. \cite{pickersgill} 
	
Pickersgill notes that “the ambiguity of psychopathy was cause for both frustration and reflection by many psychiatrists.” \cite{pickersgill} Further efforts were instituted by various physicians in order to better understand the disorder in hopes of improving treatment. Hervey Cleckley wrote The Mask of Sanity (1955) which attempted to clarify the problems associated with psychopathy and to better define the disorder. Cleckley's ideas served as a vanguard for the development of the modern understanding of the disorder. Eventually in 1952, the Diagnostic Statistical Manual of Mental Disorders (DSM) version 1 was first released, in which the disorder was referred to as Sociopathic Personality Disturbance, colloquially known as sociopathy. Sociopathy was characterized as exhibiting sexual deviance as well as “antisocial and dyssocial reactions”. \cite{pickersgill}

In 1965, child psychologist Herbert Quay came up with the notion that while all patients have a “nervous system which is sensitized to register an unfiltered input of irrelevant bits of information...the psychopathic symptomology represents the attempt on the part of the so afflicted person to minimize the sensory overload”. \cite{pickersgill} Eventually other scientists and physicians caught on and began to think about psychopathy as being rooted in a combinatino of biological and environmental causes. At this time, pscyhoanalysis was still heavily being used in order to diagnose antisocial personality disorder. By 1968, the DSM II was released and the disorder began being referred to as antisocial personality disorder. The DSM II described individuals with the disorder as “grossly selfish, callous, irresponsible, impulsive, and unable to feel guilt or to learn from experience and punishment”. \cite{pickersgill}
	
Starting in the 1990s, antisocial personality disorder was viewed from a completely different persepective, “when the pendulum is often described as having swung completely to the domain of biology and, specifically, to neuroscience and genetics”. \cite{pickersgill}In the 1970s, some psychiatrists used psychopharmacology to treat their patients. \cite{Michale2000inPickersgill} In the 1980s, the advent of technologies such as DNA sequencing and positron emission tomography (PET) allowed for more in depth research to help determine the causes of antisocial personality disorder.  With the advent of such technologies, psychoanalysis was being used less by now in order to diagnose the disorder. Furthermore, research was now being guided towards understanding which specific parts of the brain perform specific functions, which would allow researchers to understand which parts of the brain are damaged in individuals with antisocial personality disorder. 
	
The trend in people's perception of the disorder reflects the field of psychiatry's shift from understanding diseases from a “primarily psychological understanding of mental disorder to a predominantly biological model”. \cite{pickersgill}Although increasing the pool of knowledge regarding the causes of antisocial personality disorder, as well as personality disorders in general, the notion of pinpointing specific brain areas that contribute to specific aspects of a disorder is arguably problematic. The concept of mapping specific body parts to aspects of disease in general is problematic because there are usually outside forces that act in tandem with physiological factors in order to cause a diseased state in an individual. Since there exists a wide range of literature supporting the notion that environmental factors highly influence personality, defining the causes of disease should be approached cautiously. 
	
Genetic research has helped to ascertain more information implicating that the behaviors associated with antisocial personality disorder are inherited.   The gene for catechol O-methyltransferase (COMT) is known to be associated with the disorder. It was found in a study of 240 children with attention deficit hyperactivity disorder that the children with one variant of the gene "showed greater antisocial behaviors than those without this variant". \cite{ferguson} The MAOA gene is known to be associated with antisocial behavior when coupled with maltreatment in the family, meaning there is some gene-environment interaction contributing to the antisocial behavior. Since the MAOA gene is found on the X chromosome, it has been suggested that this fact may explain the higher prevalence of this disorder in males rather than females. \cite{ferguson}  The serotonin transporter promoter gene (5-HTT) is known to be associated with increased violent behavior. 

Professor Christopher Ferguson of Texas A and M University found that 56\% of the variance in antisocial personality disorder can be explained by genetic influences, 11\% can be explained by shared non-genetic influences, and 31\% can be explained by unique non-genetic influences. \cite{ferguson} Furthermore, there is evidence that specific genes may contribute to antisocial personality disorder in individuals.  There is still more work to be done on uncovering specific genes that may be associated with the disorder, mostly because the demographic of individuals with the disorder is so small that large-scale genetic studies are difficult to conduct, compared to studies of diseases such as heart disease. 
	
On the other hand, evidence exists that implicates that antisocial personality disorder can actually be acquired rather than inherited. Christina Meyers of the MD Anderson Cancer Center published a case report detailing a patient who injured the orbitofrontal lobe of his brain, and subsequently exhibited some characteristics of impaired mental function that implicated antisocial personality. In a psychological examination, the patient hallucinated about being a criminal, exhibited “impulsive, disinhibited, and tangential” behavior, and “became agitated and upset discussing certain topics”. \cite{meyers} 
	
As a result of the approach to antisocial personality disorder from a more biological perspective, various drug therapeutics have been used in attempts by physicians to combat the disorder.  Since there are several different symptoms of the disorder, several different drugs have been used.  Lithium carbonate is the best-documented medication that has been used. Lithium carbonate was originally used on prisoners, and its purpose is to reduce anger, threatening behavior and combativeness. \cite{psychcentral} Phenytoin, which has been marketed under the name Dilantin, is an anticonvulsant and has been used to reduce impulsive aggression. Some drugs that have been used to treat aggression in mentally retarded patients have also been used to psychopaths. Such drugs include propranolol, valproate, buspirone, carbamazepine, and trazodone. \cite{psychcentral}Furthermore, since mood changes are commonly associated with antisocial personality disorder, antidepressants have been used. 
	
Although drugs have been shown to be effective in treating certain symptoms of antisocial personality disorder, some physicians recommend against it because addiction can be developed for some drugs. Furthermore, after patients come off of certain drugs, their symptoms may recur. This would result in the need to constantly medicate the patients.
	
Psychological therapies have been used to treat patients, with some degree of success. Reseachers at the University of Glasgow recently reported results on a study in which they conducted cognitive behavioral therapy on 52 adult men with antisocial personality disorder. They noticed that after 12 months on the therapy, there was a significant improvement in the subjects' social interactions with others, as well as a significant decrease in the amount of problematic drinking. \cite{davidson} The fact that cognitive behavior therapy shows some improvement in patients supports the notion that there is a psychological element to the development of antisocial personality disorder, which supports the notion that both nature and nurture contribute to the disorder.
	
Surgery has been used in order to attempt to combat antisocial personality disorder. Psychosurgery has been used to alter the prefrontal cortex, which is responsible for cognition and behavior. It was reported in 1950 that cutting out parts of the prefrontal cortex that are located more dorsal (towards the patient's back) will result in greater personality deficits, making the patient “lazy, rude, boisterous, restless and inane”. \cite{freeman}On the other hand, if not enough tissue was cut out, then this would render the person more pleasant and agreeable but “lacking in tact and reserve; he may be able to retain employment, but he does not use his spare time for self-development and service to the community”. \cite{freeman}Since there are various problems such as these that could go wrong with psychosurgery, not everybody supports the usage of this treatment to treat antisocial personality disorder. Furthermore, a great amount of mental rehabilitation is required for patients who undergo psychosurgery, so it does not serve as a quick fix operation.
	
In extreme cases, individuals exhibiting antisocial psychopathic behaviors are sent to jail. They are usually sentenced to prison as a response to their behaviors, rather than for being affected by the disorder. In fact, Iowa psychiatry professor Donald Black reported that over 35\% of prison inmates have antisocial personality disorder based on a study of an Iowa prison. \cite{black} The use of prison sentences in order to combat antisocial personality disorder raises an important ethical issue. Prison has traditionally been used as a means of punishment, obtaining justice for those who are harmed by the prisoner.  Traditionally, criminals have been imprisoned. The notion of incarcerating psychopaths would naturally seem punitive. Although these prisoners committed criminal acts that resulted in incarceration, it may be argued that such criminal acts may have been committed due to their debilitating mental state due to antisocial personality disorder.
	
Antisocial personality disorder has affected up to 5\% of individuals in the American population. In the past, the public perception of the disease was more stigmatized than it is today. The change in this perception was brought about in part by the increased knowledge about the disorder, as well as further evidence for a biological basis for the disorder.  The neuroscientific solutions that have been proposed to combat the disease have helped some patients of antisocial personality disorder, however scientists are not completely confident about the efficacy of such treatments.  Ultimately, antisocial personality disorder is a complex disorder that is caused by many factors and therefore may require multiple different forms of neuroscientific solutions to treat fully. Recently, there has definitely been a great push towards treating antisocial personality disorder from a more biological perspective. Although these treatments may raise some ethical concerns, they continue to provide great results for the patients.


